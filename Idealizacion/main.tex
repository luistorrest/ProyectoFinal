\documentclass{article}
\usepackage[utf8]{inputenc}
\usepackage[spanish]{babel}
\usepackage{listings}
\usepackage{graphicx}
\graphicspath{ {images/} }
\usepackage{cite}

\begin{document}

\begin{titlepage}
    \begin{center}
        \vspace*{1cm}
            
        \Huge
        \textbf{Proyecto Final}
        
            
        \vspace{0.5cm}
        \LARGE
        Idealización 
            
        \vspace{1.5cm}
            
        \textbf{Luis Fernando Torres Torres\\Johan David Rojas Martinez}
        
        \vspace{4cm}
            
        \textbf{PhD. Augusto Salazar Jiménez}
            
        \vfill
            
        \vspace{0.8cm}
            
        \Large
        Despartamento de Ingeniería Electrónica y Telecomunicaciones\\
        Universidad de Antioquia\\
        Medellín\\
        Marzo de 2021
            
    \end{center}
\end{titlepage}

\tableofcontents%Tabla de contenidos 
\newpage

\section{Desripción principal}\label{Descripcion}
Para el desarrollo de este proyecto, se ha pensado en realizar un juego 2D con una temática infantil, resaltando algunas caricaturas muy famosas de la infancia del equipo de trabajo, las cuales pueden ser :

\begin{itemize}
\item Kid vs kat
\item Kick Butowsky
\item Phineas y Ferb 
\item Pucca 
\item Futurama
\end{itemize}

\section{Ideas} \label{ideas}
Se hizo una lluvia de ideas para así poder recopilar la mayor información e imaginación, para después definir las características finales del juego. A continuación se presenta las ideas recolectadas:

\subsection{Combinación entre caricaturas}
Esta idea está basada básicamente en mezclar diferentes personajes de las  caricaturas elegidas, respetando el rol y características originales de cada personaje, es decir; si es protagonista, antagonista, personaje secundario o extra. Lo que se busca con esta idea es integrar varios personajes en un mismo juego con el fin de tener un juego mucho mas amplio.
\subsection{Integrar espacios del desarrollo de las caricaturas }
Con esta idea lo que se busca es generar diferentes escenarios que estén contenidos en las diferentes caricaturas, respetando música, lugares conocidos, y demás objetos que caracterizan los espacios donde suceden las series propuestas.
\subsection{Juego controlado con hadware externo} 
Esta idea se realiza con el fin de poder integrar mandos externos al juego, que estén hechos a base de botones de Arduino, para poder tener una experiencia mas real a la hora de jugar, además brinda la oportunidad de tener un modo multijugador ya que una persona usaría el teclado como control y otra persona podría usar el sistema basado en Arduino.
\subsection{Cargar partida}
Esta idea esta pensada, para aquellas personas que deseen tomarse un descanso en el juego, para luego seguir jugando desde el punto donde lo dejo, el juego contará con unos puntos específicos donde se guardará la partida hasta ese lugar y así no perder todo el trabajo que se ha realizado anteriormente.
\subsection{Animaciones Llamativas}
Esta idea se trata básicamente, en realizar el juego con animaciones las cuales sean llamativas para el jugador, aplicándole figuras compactas y simétricas, colores vivos, e incluso sonidos que generen sensaciones gratificantes a la hora de jugar y además que quien lo juegue no se aburra o se esfuerce tratando de observar lo que ve en la pantalla.

\section{Realización}
Todas estas ideas estan pesnadas para realizar un juego de alta calidad,pero cabe resaltar que todas están sujetas a las limitaciones que se presenten con las herramientas que se van a usar (QTCreator,C++).

\end{document}
